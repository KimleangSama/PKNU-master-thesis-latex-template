%%%%%%%%%%%%%%%%%%%%%%%%%%%%%%%%%%%%%%%%%%%%%%%%%%%%%%%%%%%%%%%%%%%%%%%%%%%%%%%%%%%%%%%%%%%%%%%%%%%%
%
% 							PKNU Graduate thesis template in LaTeX format
%
%%%%%%%%%%%%%%%%%%%%%%%%%%%%%%%%%%%%%%%%%%%%%%%%%%%%%%%%%%%%%%%%%%%%%%%%%%%%%%%%%%%%%%%%%%%%%%%%%%%%
%%%%%%%%%%%%%%%%%%%%%%%%%%%%%%%%%%%%%%%%%%%%%%%%%%
% DOCUMENTCLASS: PKNU-thesis class
%%%%%%%%%%%%%%%%%%%%%%%%%%%%%%%%%%%%%%%%%%%%%%%%%%
% %% Options
% %%%% doctor: 박사과정 | master:  석사과정
% %%%% korean: 한글논문 | english: 영문논문
% %%%% final:  최종판   | draft:   시험판
% %%%% pdfdoc: 선택하지 않으면 북마크와 colorlink를 만들지 않습니다.(Generate bookmark and colorlink if enabled)
\documentclass[master,english,final,pdfdoc]{PKNU-thesis}



%%%%%%%%%%%%%%%%%%%%%%%%%%%%%%%%%%%%%%%%%%%%%%%%%%
% USEPACKAGE
%%%%%%%%%%%%%%%%%%%%%%%%%%%%%%%%%%%%%%%%%%%%%%%%%%
% Default package by PKNU-thesis class: geometry, indentfirst, color
% 필요에따라 \usepackage 명령어로 패키지를 이용하면 됩니다.
\usepackage{amsmath}
\usepackage{amssymb}
\usepackage{enumitem}
\usepackage{algorithm}
\usepackage{algpseudocode}
\usepackage{comment}
\usepackage{physics}
\usepackage{anyfontsize}
\newcommand{\argmax}{\operatornamewithlimits{arg\,max}}

% user custom package
\usepackage[style=phys, biblabel=brackets, block=space]{biblatex}
\usepackage{tocloft}
\usepackage{listings}

% TOC settings
\renewcommand\cftchapaftersnum{.}
\renewcommand\cftsecaftersnum{.}
\renewcommand\cftsubsecaftersnum{.}
\renewcommand\cftchapnumwidth{2.0em}
\renewcommand\cftsecnumwidth{1.4em}
\renewcommand\cftsecindent{2.0em}
\renewcommand\cftsubsecnumwidth{2.4em}
\renewcommand\cftsubsecindent{3.4em}

\renewcommand{\thechapter}{\Roman{chapter}}
\renewcommand{\thesection}{\arabic{section}}
\renewcommand{\theequation}{\arabic{chapter}.\arabic{equation}}



%%%%%%%%%%%%%%%%%%%%%%%%%%%%%%%%%%%%%%%%%%%%%%%%%%
% TITLE OF THESIS 논문제목
%%%%%%%%%%%%%%%%%%%%%%%%%%%%%%%%%%%%%%%%%%%%%%%%%%
\title[korean]{다람쥐 헌 쳇바퀴에 타고파}
\title[english]{The Quick Brown Fox Jumps Over The Lazy Dog}



%%%%%%%%%%%%%%%%%%%%%%%%%%%%%%%%%%%%%%%%%%%%%%%%%%
% AUTHOR 저자명
%%%%%%%%%%%%%%%%%%%%%%%%%%%%%%%%%%%%%%%%%%%%%%%%%%
% family_name, given_name 성, 이름을 구분해서 입력
% ex) \author[english]{family name in english}{given name in english}
% If you are a foreigner (this means you have no korean name),
% You must fill the korean name as blank, instead of deleting it or commenting it out.
\author[korean]{}{}
\author[english]{Kea}{Kimleang}


%%%%%%%%%%%%%%%%%%%%%%%%%%%%%%%%%%%%%%%%%%%%%%%%%%
% ADVISOR 지도교수
%%%%%%%%%%%%%%%%%%%%%%%%%%%%%%%%%%%%%%%%%%%%%%%%%%
% Professor name in Korean, Professor name in English, Signiture file name
\advisor[major]{Korean Profess's Name}{English Profess's Name}{nosign}


%%%%%%%%%%%%%%%%%%%%%%%%%%%%%%%%%%%%%%%%%%%%%%%%%%
% DEPARTMENT 학과명
%%%%%%%%%%%%%%%%%%%%%%%%%%%%%%%%%%%%%%%%%%%%%%%%%%
\department{AI}{engineering}


%%%%%%%%%%%%%%%%%%%%%%%%%%%%%%%%%%%%%%%%%%%%%%%%%%
% STUDENT ID
%%%%%%%%%%%%%%%%%%%%%%%%%%%%%%%%%%%%%%%%%%%%%%%%%%
% 비워있어도 무관
\studentid{2022XXXXX}



%%%%%%%%%%%%%%%%%%%%%%%%%%%%%%%%%%%%%%%%%%%%%%%%%%
% REFREEE 심사위원
%%%%%%%%%%%%%%%%%%%%%%%%%%%%%%%%%%%%%%%%%%%%%%%%%%
% %% Options [enumeration of refrees]
% %%%% Up to 5
% %% Parameters
% %%%% 한글이름(in Korean) or 영문이름(in English)
% %% 석사과정 3인, 박사과정 5인
% %% 국문논문 작성시 한글로, 영문논문 작성시 영문이름 기입
% Of course english name is available
\referee[1]{홍 상 직}
\referee[2]{옥 영 향}
\referee[3]{홍 상 직}
\referee[4]{옥 영 향}
\referee[5]{홍 상 직}



%%%%%%%%%%%%%%%%%%%%%%%%%%%%%%%%%%%%%%%%%%%%%%%%%%
% SIGNATURE OF REFREES 심사위원 서명
%%%%%%%%%%%%%%%%%%%%%%%%%%%%%%%%%%%%%%%%%%%%%%%%%%
% %% Options [enumeration of refrees]
% %%%% Up to 5
% %% Parameters
% %%%% Path of a signature figure
\refsign[1]{./figure/Erwin_Schrödinger_signature.png}
\refsign[2]{./figure/Erwin_Schrödinger_signature.png}
\refsign[3]{./figure/Erwin_Schrödinger_signature.png}
\refsign[4]{./figure/Erwin_Schrödinger_signature.png}
\refsign[5]{./figure/Erwin_Schrödinger_signature.png}



%%%%%%%%%%%%%%%%%%%%%%%%%%%%%%%%%%%%%%%%%%%%%%%%%%
% APPROVAL DATE 학위논문 승인일
%%%%%%%%%%%%%%%%%%%%%%%%%%%%%%%%%%%%%%%%%%%%%%%%%%
% %% Parameters
% %%%% Year, Month, Day 연,월,일 순으로 입력
\approvaldate{2024}{02}{28}



%%%%%%%%%%%%%%%%%%%%%%%%%%%%%%%%%%%%%%%%%%%%%%%%%%
% REFREEE DATE 심사위원 논문심사일
%%%%%%%%%%%%%%%%%%%%%%%%%%%%%%%%%%%%%%%%%%%%%%%%%%
% %% Parameters
% %%%% Year, Month, Day 연,월,일 순으로 입력
\refereedate{2024}{02}{28}



%%%%%%%%%%%%%%%%%%%%%%%%%%%%%%%%%%%%%%%%%%%%%%%%%%
% BIBLIOGRAPHY 참고문헌 (By BibLaTeX)
%%%%%%%%%%%%%%%%%%%%%%%%%%%%%%%%%%%%%%%%%%%%%%%%%%
\bibliography{thesis}


%%%%%%%%%%%%%%%%%%%%%%%%%%%%%%%%%%%%%%%%%%%%%%%%%%
% BEGINNING OF THE DOCUMENT 문서 시작
%%%%%%%%%%%%%%%%%%%%%%%%%%%%%%%%%%%%%%%%%%%%%%%%%%
\begin{document} 
    %%%%%%%%%%%%%%%%%%%%%%%%%%%%%%%%%%%%%%%%%%%%%%%%%%
    % Main Cover, Inner Cover, Approval of Statement 앞표지, 속표지, 학위논문 인준서
    %%%%%%%%%%%%%%%%%%%%%%%%%%%%%%%%%%%%%%%%%%%%%%%%%%
    % PKNU-thesis 클래스 옵션을 final로 지정해주면 자동으로 생성, draft로 지정해주면 생성되지 않습니다.

    %%%%%%%%%%%%%%%%%%%%%%%%%%%%%%%%%%%%%%%%%%%%%%%%%%
    % Table of Contents, List of Figures, List of Tables 목차, 그림목차, 표목차
    %%%%%%%%%%%%%%%%%%%%%%%%%%%%%%%%%%%%%%%%%%%%%%%%%%
    % Use \tableofcontents, \listoffigures, \listoftables
    % Or use \makecontents for showing together.
    \makecontents

    %%%%%%%%%%%%%%%%%%%%%%%%%%%%%%%%%%%%%%%%%%%%%%%%%%
    % ABSTRACT 논문요약
    %%%%%%%%%%%%%%%%%%%%%%%%%%%%%%%%%%%%%%%%%%%%%%%%%%
    % %% Parameters
    % %%%% Abstract in English, Abstract in Korean
    % 초록은 한국어논문의 경우 한국어 -> 영어 순으로, 영어논문의 경우 영어 -> 한국어로 생성됩니다.
    % 작성은 논문 종류와 관계없이 반드시 영어, 한국어 순으로 작성해주세요!
    \begin{abstract}
        {
            Abstract in English
        }
        {
            Abstract in Korean
        }
    \end{abstract}

    %%%%%%%%%%%%%%%%%%%%%%%%%%%%%%%%%%%%%%%%%%%%%%%%%%
    % Main Body 본문
    %%%%%%%%%%%%%%%%%%%%%%%%%%%%%%%%%%%%%%%%%%%%%%%%%%
    %% 이하의 본문은 LaTeX 표준 클래스 report 양식에 준하여 작성하시면 됩니다.

	\chapter{Introduction}
% \section[서론]{\hyperlink{toc}{서론}}
%%%%% PLEASE KEEP THIS LOCATION %%%%%%
\pagenumbering{arabic}
\setcounter{page}{1}
%%%%% PLEASE KEEP THIS LOCATION %%%%%%
\section{Purpose of research}
\label{subsec:intro}
Hello


	\input{./dat/02.background}
	\input{./dat/03.dataandmethod}
	\input{./dat/04.result}
	\input{./dat/05.conclusionanddiscussion}
	\input{./dat/06.appendix}

    %%%%%%%%%%%%%%%%%%%%%%%%%%%%%%%%%%%%%%%%%%%%%%%%%%
    % BIBLIOGRAPHY 참고문헌
    %%%%%%%%%%%%%%%%%%%%%%%%%%%%%%%%%%%%%%%%%%%%%%%%%%
    % For BibLaTeX
    \addcontentsline{toc}{chapter}{Bibliography}
    \printbibliography

    % For BibTeX / natbib
    % \let\Section\section
    % \def\section*#1{\Section*{{\hyperlink{toc}{#1}}}}
    % \renewcommand{\thesection}{}
    % \bibliographystyle{chicago}
    % \bibliography{thesis}
    % \nocite{apsrev42Control}
    % \bibliographystyle{./revtex4-2_bst/apsrev4-2}
    % \bibliography{thesis}
    % Copy and Paste the code in .bib file below
    % @CONTROL{REVTEX42Control}
    % @CONTROL{apsrev42Control,author="00",editor="1",pages="1",title="0",year="0"}

    %%%%%%%%%%%%%%%%%%%%%%%%%%%%%%%%%%%%%%%%%%%%%%%%%%
    % ACKNOWLEDGEMENT 감사의 글
    %%%%%%%%%%%%%%%%%%%%%%%%%%%%%%%%%%%%%%%%%%%%%%%%%%
    % %% Options 
    % %%%% 1: in Korean 2: in English
    \acknowledgement[2]
    Don't be afraid to give up the good to go for the great. - John D. Rockfeller

%%%%%%%%%%%%%%%%%%%%%%%%%%%%%%%%%%%%%%%%%%%%%%%%%%
% END OF THE DOCUMENT 문서 끝
%%%%%%%%%%%%%%%%%%%%%%%%%%%%%%%%%%%%%%%%%%%%%%%%%%
\end{document}